\documentclass{article}
\begin{document}
\subsection{SensorAPI}
Bosch Sensortec recommends using the SensorAPI in order to communicate with the sensors. The SensorAPI, an abstraction layer written in C makes it much more convenient for the user to access the register map of the sensor, in order to configure certain functionality and obtain certain information from it.

For making use of the SensorAPI, some function pointers must be set to the appropriate read/write functions of the selected bus on the system (either I\textsuperscript{2}C or SPI), as well as one function pointer to a system's function causing delays in milliseconds.

In order to execute C code using SensorAPI, the COINES API provides the mentioned read, write, delay functions. These functions are wrapper functions, embedding the actual SensorAPI payloads into a transport package, sending this via USB or BLE to the Engineering board, where the payload is translated into corresponding SPI or I\textsuperscript{2}C messages and sent to the sensor on the shuttle board.The mapping would look similar to the one below.

\begin{lstlisting}
#include "coines.h"
#include "bst_sensor.h"

struct bst_sensor_dev sensordev;
....
....
sensordev.intf = BST_SENSOR_I2C_INTF;  // SPI - BST_SENSOR_SPI_INTF
sensordev.read = coines_read_i2c;   // coines_read_spi
sensordev.write = coines_write_i2c; // coines_write_spi
sensordev.delay_ms = coines_delay_usec;

\end{lstlisting}

For the description of COINES functions used, refer to \ref{CoinesCFunctions}.


\end{document}