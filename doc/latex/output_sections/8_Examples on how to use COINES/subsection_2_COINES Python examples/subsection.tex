\documentclass{article}
\begin{document}
\subsection{COINES Python examples}

\subsubsection{Getting board info}\label{GettingBoardInfo}
The following code snippet shows how to get board information.
\begin{lstlisting}[language=python]
	import coinespy as cpy
	from coinespy import ErrorCodes
	
	COM_INTF = cpy.CommInterface.USB
	
	if __name__ == "__main__":
		board = cpy.CoinesBoard()
		print('coinespy version - %s' % cpy.__version__)
		board.open_comm_interface(COM_INTF)
		if board.error_code != ErrorCodes.COINES_SUCCESS:
			print(f'Could not connect to board: {board.error_code}')
		else:
			b_info = board.get_board_info()
			print(f"coines lib version: {board.lib_version}")
			print(
				f'BoardInfo: HW/SW ID: {hex(b_info.HardwareId)}/{hex(b_info.SoftwareId)}')
			board.close_comm_interface()
\end{lstlisting}

\subsubsection{I2C config and read}
This basic program shows how to configure and perform I2C read.
\newline Sensor: BMI085
\begin{lstlisting}[language=python]
	import sys
	import time
	import coinespy as cpy
	from coinespy import ErrorCodes

	COM_INTF = cpy.CommInterface.USB
	
	if __name__ == "__main__":
		BOARD = cpy.CoinesBoard()

		BOARD.open_comm_interface(COM_INTF)
		if BOARD.error_code != ErrorCodes.COINES_SUCCESS:
			print(f"Open Communication interface: {BOARD.error_code}")
			sys.exit()
	
		BMI085_I2C_ADDRESS_ACCEL = 0x18
		BMI085_I2C_ADDRESS_GYRO  = 0x68
		BMI08_REG_ACCEL_CHIP_ID = 0x00
	
		BOARD.set_shuttleboard_vdd_vddio_config(vdd_val=0, vddio_val=0)
	
		#  Config I2C pins
		BOARD.set_pin_config(
			cpy.MultiIOPin.SHUTTLE_PIN_8, cpy.PinDirection.OUTPUT, cpy.PinValue.LOW)
		BOARD.set_pin_config(
			cpy.MultiIOPin.SHUTTLE_PIN_SDO, cpy.PinDirection.OUTPUT, cpy.PinValue.LOW)
		# Set PS pin of gyro to HIGH for proper protocol selection
		BOARD.set_pin_config(
			cpy.MultiIOPin.SHUTTLE_PIN_9, cpy.PinDirection.OUTPUT, cpy.PinValue.HIGH)
	
		# I2C config
		BOARD.config_i2c_bus(
			cpy.I2CBus.BUS_I2C_0, BMI085_I2C_ADDRESS_ACCEL, cpy.I2CMode.STANDARD_MODE)
	
		BOARD.set_shuttleboard_vdd_vddio_config(vdd_val=3.3, vddio_val=3.3)
		time.sleep(0.2)
	
		# I2C read
		accel_chip_id = BOARD.read_i2c(
			cpy.I2CBus.BUS_I2C_0, BMI08_REG_ACCEL_CHIP_ID, 1, BMI085_I2C_ADDRESS_ACCEL)
		gyro_chip_id = BOARD.read_i2c(
			cpy.I2CBus.BUS_I2C_0, BMI08_REG_ACCEL_CHIP_ID, 1, BMI085_I2C_ADDRESS_GYRO)
	
		print(f"Accel chip id: {hex(accel_chip_id[0])}")
		print(f"Gyro chip id: {hex(gyro_chip_id[0])}")
	
		# Deinit board
		BOARD.set_shuttleboard_vdd_vddio_config(vdd_val=0, vddio_val=0)
		BOARD.soft_reset()
	
		BOARD.close_comm_interface()	
\end{lstlisting}
The user shall pass GPIO pin numbers, read register address and I2C device address for sensors based on the selected sensor shuttle board. I2C communication require the proper setting of VDD and VDDIO using \texttt{set\_shuttleboard\_vdd\_vddio\_config}.

\subsubsection{SPI config and read}
This basic program shows how to configure and perform SPI read.
\newline Sensor: BMI085
\begin{lstlisting}[language=python]
	import sys
	import time
	import coinespy as cpy
	from coinespy import ErrorCodes
	
	COM_INTF = cpy.CommInterface.USB
	
	if __name__ == "__main__":
		BOARD = cpy.CoinesBoard()

		BOARD.open_comm_interface(COM_INTF)
		if BOARD.error_code != ErrorCodes.COINES_SUCCESS:
			print(f"Open Communication interface: {BOARD.error_code}")
			sys.exit()
	
		BMI085_ACCEL_CS_PIN = cpy.MultiIOPin.SHUTTLE_PIN_8
		BMI085_GYRO_CS_PIN = cpy.MultiIOPin.SHUTTLE_PIN_14
		BMI08_REG_ACCEL_CHIP_ID = 0x00
		accel_dummy_byte_len = 1
	
		BOARD.set_shuttleboard_vdd_vddio_config(vdd_val=0, vddio_val=0)
	
		# Config CS pin
		BOARD.set_pin_config(
			BMI085_ACCEL_CS_PIN, cpy.PinDirection.OUTPUT, cpy.PinValue.HIGH)
		BOARD.set_pin_config(
			BMI085_GYRO_CS_PIN, cpy.PinDirection.OUTPUT, cpy.PinValue.HIGH)
		# Set PS pin of gyro to LOW for proper protocol selection
		BOARD.set_pin_config(
			cpy.MultiIOPin.SHUTTLE_PIN_9, cpy.PinDirection.OUTPUT, cpy.PinValue.LOW)
	
		#  SPI config
		BOARD.config_spi_bus(cpy.SPIBus.BUS_SPI_0, BMI085_ACCEL_CS_PIN,
							 cpy.SPISpeed.SPI_1_MHZ, cpy.SPIMode.MODE0)
	
		BOARD.set_shuttleboard_vdd_vddio_config(vdd_val=3.3, vddio_val=3.3)
		time.sleep(0.2)
	
		# Initialize SPI by dummy read
		reg_data = BOARD.read_spi(cpy.SPIBus.BUS_SPI_0, BMI08_REG_ACCEL_CHIP_ID, 1)
	
		# SPI read
		accel_chip_id = BOARD.read_spi(
			cpy.SPIBus.BUS_SPI_0, BMI08_REG_ACCEL_CHIP_ID, 1 + accel_dummy_byte_len, BMI085_ACCEL_CS_PIN)
		gyro_chip_id = BOARD.read_spi(
			cpy.SPIBus.BUS_SPI_0, BMI08_REG_ACCEL_CHIP_ID, 1, BMI085_GYRO_CS_PIN)
	
		print(f"Accel chip id: {hex(accel_chip_id[accel_dummy_byte_len])}")
		print(f"Gyro chip id: {hex(gyro_chip_id[0])}")
	
		# Deinit board
		BOARD.set_shuttleboard_vdd_vddio_config(vdd_val=0, vddio_val=0)
		BOARD.soft_reset()
	
		BOARD.close_comm_interface()
\end{lstlisting}
The user shall pass GPIO pin numbers, read register address and SPI CS pins for sensors based on the selected sensor shuttle board. SPI communication require the proper setting of VDD and VDDIO using \texttt{set\_shuttleboard\_vdd\_vddio\_config}.

\newpage

\end{document}