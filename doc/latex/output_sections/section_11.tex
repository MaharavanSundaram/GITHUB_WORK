\documentclass{article}
\begin{document}
\section{USB/BLE DFU bootloader}

A USB/BLE  Bootloader for APP3.x/nRF52840 and Nicla Sense ME/nRF52832 chip comply with below items:

\begin{itemize}
	\item \url{https://www.usb.org/sites/default/files/DFU_1.1.pdf}
	\item \href{https://infocenter.nordicsemi.com/index.jsp?topic=%2Fcom.nordic.infocenter.sdk5.v15.2.0%2Fble_sdk_app_dfu_bootloader.html&cp=9_5_3_4_1_3}{nRF5 SDK v15.2.0 - BLE Secure DFU Bootloader}
\end{itemize}

\subsection{Key Features}

\subsubsection{USB DFU}
The key features of USB DFU are as follows:
\begin{itemize}
	\item Code download to RAM or FLASH
	\item Code read back (upload) from  RAM or FLASH (Useful for taking firmware backups)
	\item Works with Windows, Linux, macOS and Android.
\end{itemize}

\subsubsection{BLE DFU}
The key features of BLE DFU are as follows:
\begin{itemize}
	\item Code download to FLASH.
	\item Works with PC and mobile devices with iOS/Android.
\end{itemize}

Bootloader was written taking into account the following aspects:
\begin{itemize}
	\item Usability.
	\begin{enumerate}[label=\roman*.]
		\item No special driver installation or admin rights should be required.
		\item The update process should be straight forward.
	\end{enumerate}
	\item Maintainability
	\begin{enumerate}[label=\roman*.]
		\item Open source community takes care of PC side tools. For eg: dfu-util is a cross platform tool.
		\item Use Google Chrome's WebUSB to update firmware. Sample implementation \url{https://devanlai.github.io/webdfu/dfu-util/}
	\end{enumerate}
	\item Size
	\item COINES on MCU.
\end{itemize}

\subsection{Invoking the Bootloader}
\begin{enumerate}
	\item To invoke Bootloader from Hardware, switch the board to bootloader mode (refer to section \ref{SwitchModes}).
	\item To invoke Bootloader from Software, use the below snippets in your program based on the board selected.
\end{enumerate}

\begin{itemize}
	\item APP3.x
	\begin{enumerate}[label=\roman*.]
		\item Write 0x4E494F43 ('N','I','O','C') to MAGIC\_LOCATION (0x2003FFF4)
		\item Write 0x0 or 0xF0000 to APP\_START\_ADDR (0x2003FFF8)
		\item Call NVIC\_SystemReset()
		\begin{lstlisting}[language=C]
		
		#define  MAGIC_LOCATION (0x2003FFF4)
		#define  APP_START_ADDR (*(uint32_t *)(MAGIC_LOCATION+4)
		 
		*((uint32_t *)MAGIC_LOCATION) == 0x4E494F43;
		APP_START_ADDR = 0xF0000;
		//APP_START_ADDR = 0x0;
		NVIC_SystemReset();
		
		\end{lstlisting}
	\end{enumerate}
\end{itemize}

\begin{itemize}
	\item Nicla Sense ME Board
	\begin{enumerate}[label=\roman*.]
		\item Write 0x544F4F42 ('T','O','O','B') to MAGIC\_LOCATION (0x2000F804)
		\item Call NVIC\_SystemReset()
		\begin{lstlisting}[language=C]
		
		#define  MAGIC_LOCATION (0x2000F804)
		#define  APP_START_ADDR (*(uint32_t *)(MAGIC_LOCATION+4)
		 
		*((uint32_t *)MAGIC_LOCATION) == 0x544F4F42;
		NVIC_SystemReset();
		
		\end{lstlisting}
	\end{enumerate}
\end{itemize}
It is to be noted that the same feature can also be used to perform application switch ( 2 or more applications can reside in the same flash memory at different address locations ). Just write the application start address to APP\_START\_ADDR instead of bootloader address
\subsection{Using the Bootloader via USB}
The commands below demonstrate how to use dfu-util for different scenarios:

\begin{itemize}
	\item Path to dfu-util: \path{\tools\usb-dfu}
\end{itemize}

Write firmware to Flash memory using following command
\begin{itemize}
	\item dfu-util -a FLASH -D <firmware>.bin -R
\end{itemize}

Write firmware to RAM memory using following command
\begin{itemize}
	\item dfu-util -a RAM -D <firmware>.bin -R
\end{itemize}

Read firmware from Flash memory using following command
\begin{itemize}
	\item dfu-util -a FLASH -U <firmware>.bin
\end{itemize}

Read firmware from RAM memory using following command
\begin{itemize}
	\item dfu-util -a RAM -U <firmware>.bin
\end{itemize}

Read device serial number/ BLE MAC address
\begin{itemize}
	\item dfu-util -l
	
Note: Not applicable for Nicla Sense ME board
\end{itemize}

\subsection{Using the Bootloader via BLE}
To update the bootloader firmware via BLE, proceed as follows:
\begin{itemize}
	\item PC (Windows, Linux or macOS)
	\newline Python script present in following path \path{\tools\app30-ble-dfu} can use the binary file directly.
	\begin{enumerate}[label=\roman*.]
		\item Refer to section \ref{SwitchModes} to switch to Bootloader mode
		\item Run the command:
		\begin{itemize}
			\item \texttt{pip install -r requirements.txt}
		\end{itemize} 
		\item Scan for devices to find BLE MAC address using below command
		\begin{itemize}
			\item \texttt{python app30-ble-dfu.py -l}
		\end{itemize} 
		\item Update firmware by using MAC address obtained in the previous step and firmware BIN file
		\begin{itemize}
			\item \texttt{python app30-ble-dfu.py -d D7:A3:CE:8E:36:14 -f <firmware>.bin}
		\end{itemize}
	\end{enumerate}
	\item Android devices
	\begin{enumerate}[label=\roman*.]
		\item Generate ZIP package using \url{https://pypi.org/project/adafruit-nrfutil/} before using nRF ToolBox for BLE or nRF connect for mobile.
		\begin{itemize}
			\item \texttt{adafruit-nrfutil dfu genpkg --dev-type 0x0052 --application <firmware>.bin dfu-package.zip}
		\end{itemize} 
	\end{enumerate}
Note: Not applicable for Nicla Sense ME board
\end{itemize}

\newpage


\end{document}