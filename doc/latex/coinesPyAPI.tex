\subsection{COINES Python functions}

As coinespy is only a wrapper on top of coinesAPI, the following API documentation is limited to the wrapper only. Details about meaning of variables and functionality can be found in the corresponding coinesAPI documentation in the chapter above.
\vspace{12pt}
The following function calls are defined within the class \texttt{CoinesBoard}. Thus in order to access the functions, the user has to create an object of that class first.

\begin{lstlisting}[language=python]
	import coinespy as cpy
	coinesboard = cpy.CoinesBoard()
\end{lstlisting}

\subsubsection{coinespy API calls: Interface and board information}

\paragraph{open\_comm\_interface}
Sets the communication interface between board and PC to USB, Serial or BLE.

\begin{lstlisting}[language=python]
coinesboard.open_comm_interface(interface=CommInterface.USB, serial_com_config: SerialComConfig = None,
ble_com_config: BleComConfig = None) -> ErrorCodes
\end{lstlisting}

For the definition of \texttt{CommInterface}, refer to \ref{CommInterface}.

\paragraph{close\_comm\_interface}
Disposes the resources used by the USB/serial/BLE communication.

\begin{lstlisting}[language=python]
coinesboard.close_comm_interface(arg=None) -> ErrorCodes
\end{lstlisting}

\paragraph{get\_board\_info}
Obtains board specific information.

\begin{lstlisting}[language=python]
BoardInfo = coinesboard.get_board_info()

# Return:
BoardInfo.HardwareId    # Hardware ID
BoardInfo.SoftwareId    # Firmware version information
BoardInfo.Board         # Board type
BoardInfo.ShuttleID     # ID of shuttle, in case a shuttle is detected
\end{lstlisting}

\paragraph{scan\_ble\_devices}
This API is used to connect to BLE Adapter and return list of BLE peripherals found during BLE scan.

\begin{lstlisting}
ble_info, peripheral_count  = coinesboard.scan_ble_devices(scan_timeout_ms=0) -> Tuple[list, int]
\end{lstlisting}

For the definition of parameters, refer to \ref{coinesScanBleDevices}.

\paragraph{echo\_test}
This API is used to test the communication.

\begin{lstlisting}
coinesboard.echo_test(data: List[int]) -> ErrorCodes
\end{lstlisting}

Arguments:
\begin{itemize}
	\item \texttt{data}: Data to be sent for testing.
\end{itemize}

\subsubsection{coinespy API calls: GPIO oriented calls}

\paragraph{set\_pin\_config}
Configures the state, level and direction of a GPIO pin

\begin{lstlisting}[language=python]
coinesboard.set_pin_config(pin_number: MultiIOPin, direction: PinDirection, output_state: PinValue) -> ErrorCodes
\end{lstlisting}

For the definition of \texttt{MultiIOPin}, refer to \ref{MultiIOPin}. For the definition of \texttt{PinDirection}, refer to \ref{PinDirection}.
For \texttt{PinValue}, refer to \ref{PinValue}.

\paragraph{get\_pin\_config}
Obtains information regarding the Pin's state, level and direction.

\begin{lstlisting}[language=python]
PinConfigInfo = coinesboard.get_pin_config(pin_number: MultiIOPin)

# Return:
PinConfigInfo.direction         # 0: INPUT, 1: OUTPUT
PinConfigInfo.switch_state       # 0: OFF, 1: ON
PinConfigInfo.level             # 1: HIGH, 0: LOW
\end{lstlisting}

\paragraph{set\_shuttleboard\_vdd\_vddio\_config}

Set the VDD and VDDIO voltage level.

\begin{lstlisting}
coinesboard.set_shuttleboard_vdd_vddio_config(vdd_val: float = None, vddio_val: float = None) -> ErrorCodes

# Example: coinesboard.set_shuttleboard_vdd_vddio_config(3.3, 3.3)
\end{lstlisting}

\paragraph{set\_vdd}

Set the VDD voltage level.

\begin{lstlisting}
coinesboard.set_vdd(vdd_val: float = None) -> ErrorCodes

# Example: coinesboard.set_vdd(3.3)
\end{lstlisting}

\paragraph{set\_vddio}

Set the VDDIO voltage level.

\begin{lstlisting}
coinesboard.set_vddio(vdd_val: float = None) -> ErrorCodes

# Example: coinesboard.set_vddio(3.3)
\end{lstlisting}

\subsubsection{coinespy API calls: Sensor communication}

For the definition of \texttt{SPIBus}, refer to \ref{SPIBus}. For the definition of \texttt{I2CBus}, refer to \ref{I2CBus}.

\paragraph{config\_i2c\_bus}
Configures the I\textsuperscript{2}C bus.

\begin{lstlisting}
coinesboard.config_i2c_bus(bus: I2CBus, i2c_address: int, i2c_mode: I2CMode) -> ErrorCodes
\end{lstlisting}

For the definition of \texttt{I2CMode}, refer to \ref{I2CMode}.

\paragraph{config\_spi\_bus}
Configures the SPI bus of the board.
\begin{lstlisting}
coinesboard.config_spi_bus(bus: SPIBus, cs_pin: MultiIOPin, spi_speed=SPISpeed, spi_mode=SPIMode) -> ErrorCodes
\end{lstlisting}

For the definition of \texttt{MultiIOPin}, refer to \ref{MultiIOPin}. For the definition of \texttt{SPISpeed}, refer to \ref{SPISpeed}. For the definition of \texttt{SPIMode}, refer to \ref{SPIMode}.

\paragraph{deconfig\_i2c\_bus}
This API is used to de-configure the I\textsuperscript{2}C bus

\begin{lstlisting}
coinesboard.deconfig_i2c_bus(bus: I2CBus) -> ErrorCodes
\end{lstlisting}

\paragraph{deconfig\_spi\_bus}
This API is used to de-configure the SPI bus

\begin{lstlisting}
coinesboard.deconfig_spi_bus(bus: SPIBus) -> ErrorCodes
\end{lstlisting}

\paragraph{write\_i2c}
Writes 8-bit register data to the I\textsuperscript{2}C

\begin{lstlisting}
coinesboard.write_i2c(bus: I2CBus, register_address: int, register_value: int, sensor_interface_detail: int = None) -> ErrorCodes
\end{lstlisting}

For the definition of parameters, refer to \ref{CoinesWriteI2c}.

\paragraph{read\_i2c}
Reads 8-bit register data from the I\textsuperscript{2}C

\begin{lstlisting}
register_data = coinesboard.read_i2c(bus: I2CBus, register_address: int, number_of_reads=1, sensor_interface_detail: int = None)
\end{lstlisting}

For the definition of parameters, refer to \ref{CoinesReadI2c}.

\paragraph{write\_spi}
Writes 8-bit register data to the SPI device

\begin{lstlisting}
coinesboard.write_spi(bus: SPIBus, register_address: int, register_value: int, sensor_interface_detail: int = None) -> ErrorCodes
\end{lstlisting}


For the definition of parameters, refer to \ref{CoinesWriteSpi}.

\paragraph{read\_spi}
Reads 8-bit register data from the SPI device.

\begin{lstlisting}
register_data = coinesboard.read_spi(bus: SPIBus, register_address: int, number_of_reads=1, sensor_interface_detail: int = None)
\end{lstlisting}

For the definition of parameters, refer to \ref{CoinesReadSpi}.

\paragraph{delay\_milli\_sec}
Introduces delay in millisecond.

\begin{lstlisting}
coinesboard.delay_milli_sec(time_in_milli_sec=100)
\end{lstlisting}

\paragraph{delay\_micro\_sec}
Introduces delay in microsecond.

\begin{lstlisting}
coinesboard.delay_micro_sec(time_in_micro_sec=1)
\end{lstlisting}

\subsubsection{coinespy API calls: Streaming feature}

\paragraph{config\_streaming}

Sets the configuration for streaming sensor data.

\begin{lstlisting}[language=python]
coinesboard.config_streaming(sensor_id: int,
	stream_config: StreamingConfig, data_blocks: StreamingBlocks) -> ErrorCodes
\end{lstlisting}

Arguments:
\begin{itemize}
	\item \texttt{sensor\_id}: An integer number that can be used as identifier/index to the sensor data that will be streamed for this setting

	\item \texttt{stream\_config}: Contains information regarding interface settings and streaming configuration.
	\item  \texttt{data\_blocks}: Contains information regarding numbers of blocks to read, register address and size for each block.
\end{itemize}
Note:\newline
The below parameters should always be set:
\begin{itemize}
	\item \texttt{data\_blocks.NoOfBlocks}: number of blocks to stream (must at least be one)
	\item For each block b:
	      \begin{itemize}
		      \item \texttt{data\_blocks.RegStartAddr[b]}: start address of the block in the register map
		      \item \texttt{data\_blocks.NoOfDataBytes[b]}: number of bytes to read, starting from the start address
	      \end{itemize}
\end{itemize}
\vspace{12pt}
For reading data from I\textsuperscript{2}C bus,then set the below parameters:
\begin{itemize}
	\item \texttt{stream\_config.Intf = cpy.SensorInterface.I2C.value}
	\item \texttt{stream\_config.I2CBus}: I\textsuperscript{2}C bus (in case of APP2.0 and APP3.x, this is always\\
	      \texttt{cpy.I2CBus.BUS\_I2C\_0.value})
	\item \texttt{stream\_config.DevAddr}: I\textsuperscript{2}C address of the sensor
\end{itemize}
\vspace{12pt}
For reading data from SPI bus, then set the below parameters:
\begin{itemize}
	\item \texttt{stream\_config.Intf = cpy.SensorInterface.SPI.value;}
	\item \texttt{stream\_config.SPIBus}: SPI bus (in case of APP2.0 and APP3.x, this is always\\
	      \texttt{cpy.SPIBus.BUS\_SPI\_0.value})
	\item \texttt{stream\_config.CSPin}: CS pin of the sensor, information can be obtained from the shuttle board documentation for the sensor.
	\item \texttt{stream\_config.SPIType}: 0 : 8-bit SPI; 1 : 16-bit SPI
\end{itemize}
\vspace{12pt}
When polling mode is requested, set the below parameters:
\begin{itemize}
	\item \texttt{stream\_config.SamplingUnits}: either milliseconds or microseconds. Refer to \ref{SamplingUnits}.
	\item \texttt{stream\_config.SamplingTime}: sampling period in the unit as defined in \\ \texttt{stream\_config.SamplingUnits}
\end{itemize}
\vspace{12pt}
When interrupt mode is requested, set the below parameters:
\begin{itemize}
	\item \texttt{stream\_config.IntPin}: pin of the interrupt which shall trigger the sensor read-out. If the interrupt output of the sensor is used, the required information about the pin number can be obtained from the shuttle board documentation for the sensor.
	\item \texttt{stream\_config.IntTimeStamp}:  it can be configured if the sensor data is tagged with a timestamp - 1 or not - 0.
	\item \texttt{stream\_config.HwPinState}: State of the hardware pin connected to the interrupt line - 0/1 : Low/high
\end{itemize}
\vspace{12pt}
Below parameters are common for both streaming types:
\begin{itemize}
	\item \texttt{stream\_config.IntlineCount}: Number of interrupt lines to be used for monitoring interrupts.
	\item \texttt{stream\_config.IntlineInfo}: List of pin numbers that correspond to interrupt lines being used for interrupt monitoring.
	\item \texttt{stream\_config.ClearOnWrite}: 0/1 : Disable/enable "clear on write" feature
\end{itemize}
\vspace{12pt}
The below parameters should be set only when stream\_config.ClearOnWrite = 1:
\begin{itemize}
	\item \texttt{stream\_config.ClearOnWriteConfig.StartAddress}: Address of the sensor register at which the process of clearOnWrite should initiate.
	\item \texttt{stream\_config.ClearOnWriteConfig.DummyByte}: Number of padding bytes that must be added before clearing the bytes starting from the designated address.
	\item \texttt{stream\_config.ClearOnWriteConfig.NumBytesToClear}: Number of bytes that need to be cleared.
\end{itemize}
\vspace{12pt}
Below is the Python code snippet for interrupt streaming
\begin{lstlisting}[language=Python]
# Store streaming settings in local variables
accel_stream_settings = dict(
	I2C_ADDR_PRIMARY=0x18,
	NO_OF_BLOCKS = 2,
	REG_X_LSB= [0x12, 0x00],
	NO_OF_DATA_BYTES= [6, 1],
	CHANNEL_ID=1,
	CS_PIN=cpy.MultiIOPin.SHUTTLE_PIN_8.value,
	INT_PIN=cpy.MultiIOPin.SHUTTLE_PIN_21.value,
	INT_TIME_STAMP=1,
)
gyro_stream_settings = dict(
	I2C_ADDR_PRIMARY=0x68,
	NO_OF_BLOCKS = 2,
	REG_X_LSB= [0x02,0x00],
	NO_OF_DATA_BYTES = [6, 1],
	CHANNEL_ID=2,
	CS_PIN=cpy.MultiIOPin.SHUTTLE_PIN_14.value,
	INT_PIN=cpy.MultiIOPin.SHUTTLE_PIN_22.value,
	INT_TIME_STAMP=1,
)


# set the config_streaming parameters
stream_config = cpy.StreamingConfig()
data_blocks = cpy.StreamingBlocks()
if self.interface == cpy.SensorInterface.I2C:
	stream_config.Intf = cpy.SensorInterface.I2C.value
	stream_config.I2CBus = cpy.I2CBus.BUS_I2C_0.value
	stream_config.DevAddr = sensor["I2C_ADDR_PRIMARY"]

elif self.interface == cpy.SensorInterface.SPI:
	stream_config.Intf = cpy.SensorInterface.SPI.value
	stream_config.SPIBus = cpy.SPIBus.BUS_SPI_0.value
	stream_config.CSPin = sensor["CS_PIN"]

if sensor_type == bmi08x.SensorType.ACCEL and self.interface == cpy.SensorInterface.SPI:
	# extra dummy byte for SPI
	dummy_byte_offset = 1
else:
	dummy_byte_offset = 0

data_blocks.NoOfBlocks = sensor["NO_OF_BLOCKS"]
for i in range(0, data_blocks.NoOfBlocks):
	data_blocks.RegStartAddr[i] = sensor["REG_X_LSB"][i]
	data_blocks.NoOfDataBytes[i] = sensor["NO_OF_DATA_BYTES"][i] + dummy_byte_offset

stream_config.IntTimeStamp = sensor["INT_TIME_STAMP"]
stream_config.IntPin = sensor["INT_PIN"]

# call config_streaming API for each sensor to configure the streaming settings
ret = coinesboard.config_streaming(
	accel_sensor_id, self.accel_stream_config, self.accel_data_blocks)
ret = coinesboard.config_streaming(
	gyro_sensor_id, self.accel_stream_config, self.accel_data_blocks)

\end{lstlisting}



\paragraph{start\_stop\_streaming}

Starts or stops sensor data streaming.

\begin{lstlisting}[language=python]
coinesboard.start_stop_streaming(stream_mode: StreamingMode, start_stop: StreamingState) -> ErrorCodes
\end{lstlisting}

For the definition of \texttt{StreamingMode}, refer to \ref{StreamingMode}. For the definition of \texttt{StreamingState}, refer to \ref{StreamingState}.

\paragraph{read\_stream\_sensor\_data}

Reads the data streamed from the sensor.

\begin{lstlisting}[language=python]
coinesboard.read_stream_sensor_data(sensor_id: int, number_of_samples: int,
	buffer_size=STREAM_RSP_BUF_SIZE) -> Tuple[ErrorCodes, list, int]
\end{lstlisting}

Return:\\
Tuple of ErrorCodes, data and valid\_samples\_count\newline
For the detailed definition of parameters, refer to \ref{coinesReadStreamSensorData}.

\subsubsection{coinespy API calls: Other useful APIs}

\paragraph{flush\_interface}
Flush the write buffer.

\begin{lstlisting}
coinesboard.flush_interface()
\end{lstlisting}

\paragraph{soft\_reset}
Resets the device.

\begin{lstlisting}
coinesboard.soft_reset()
\end{lstlisting}

\subsubsection{Definition of constants}

\paragraph{PinDirection}\label{PinDirection}

Pin mode definitions

\begin{lstlisting}
class PinDirection:
    INPUT = 0  # COINES_PIN_DIRECTION_IN = 0
    OUTPUT = 1
\end{lstlisting}

\paragraph{PinValue}\label{PinValue}

Pin level definitions

\begin{lstlisting}
class PinValue:
    LOW = 0  # COINES_PIN_VALUE_LOW = 0
    HIGH = 1
\end{lstlisting}

\paragraph{CommInterface}\label{CommInterface}

Definition of Communication interface

\begin{lstlisting}
class CommInterface:
    USB = 0
    SERIAL = 1
    BLE = 2
\end{lstlisting}

\paragraph{I2CMode}\label{I2CMode}

Definition of the speed of I2C bus.

\begin{lstlisting}
class I2CMode:
	STANDARD_MODE = 0 # Standard mode - 100kHz
	FAST_MODE = 1 # Fast mode - 400kHz
	SPEED_3_4_MHZ = 2 # High Speed mode - 3.4 MHz
	SPEED_1_7_MHZ = 3 # High Speed mode 2 - 1.7 MHz
\end{lstlisting}

\paragraph{SPISpeed}\label{SPISpeed}

Definition of the speed of SPI bus.

\begin{lstlisting}
class SPISpeed:
	SPI_10_MHZ = 6
	SPI_7_5_MHZ = 8
	SPI_6_MHZ = 10
	SPI_5_MHZ = 12
	SPI_3_75_MHZ = 16
	SPI_3_MHZ = 20
	SPI_2_5_MHZ = 24
	SPI_2_MHZ = 30
	SPI_1_5_MHZ = 40
	SPI_1_25_MHZ = 48
	SPI_1_2_MHZ = 50
	SPI_1_MHZ = 60
	SPI_750_KHZ = 80
	SPI_600_KHZ = 100
	SPI_500_KHZ = 120
	SPI_400_KHZ = 150
	SPI_300_KHZ = 200
	SPI_250_KHZ = 240
\end{lstlisting}

\paragraph{SPITransferBits}\label{SPITransferBits}

Definition of the SPI bits.

\begin{lstlisting}
class SPITransferBits:
	SPI8BIT = 8 # 8 bit register read/write
	SPI16BIT = 16 # 16 bit register read/write
\end{lstlisting}

\paragraph{SPIMode}\label{SPIMode}

Definition of the SPI mode.

\begin{lstlisting}
class SPIMode:
	MODE0 = 0x00 # SPI Mode 0: CPOL=0; CPHA=0
	MODE1 = 0x01 # SPI Mode 1: CPOL=0; CPHA=1
	MODE2 = 0x02 # SPI Mode 2: CPOL=1; CPHA=0
	MODE3 = 0x03 # SPI Mode 3: CPOL=1; CPHA=1
\end{lstlisting}

\paragraph{MultiIOPin}\label{MultiIOPin}

Definition of the shuttle board pin(s)

\begin{lstlisting}
class MultiIOPin(Enum):
	SHUTTLE_PIN_7 = 0x09 # CS pin
	SHUTTLE_PIN_8 = 0x05 # Multi-IO 5
	SHUTTLE_PIN_9 = 0x00 # Multi-IO 0
	SHUTTLE_PIN_14 = 0x01 # Multi-IO 1
	SHUTTLE_PIN_15 = 0x02 # Multi-IO 2
	SHUTTLE_PIN_16 = 0x03 # Multi-IO 3
	SHUTTLE_PIN_19 = 0x08 # Multi-IO 8
	SHUTTLE_PIN_20 = 0x06 # Multi-IO 6
	SHUTTLE_PIN_21 = 0x07 # Multi-IO 7
	SHUTTLE_PIN_22 = 0x04 # Multi-IO 4
	SHUTTLE_PIN_SDO = 0x1F

	# APP3.x pins
	MINI_SHUTTLE_PIN_1_4 = 0x10  # GPIO0
	MINI_SHUTTLE_PIN_1_5 = 0x11  # GPIO1
	MINI_SHUTTLE_PIN_1_6 = 0x12  # GPIO2/INT1
	MINI_SHUTTLE_PIN_1_7 = 0x13  # GPIO3/INT2
	MINI_SHUTTLE_PIN_2_5 = 0x14  # GPIO4
	MINI_SHUTTLE_PIN_2_6 = 0x15  # GPIO5
	MINI_SHUTTLE_PIN_2_1 = 0x16  # CS
	MINI_SHUTTLE_PIN_2_3 = 0x17  # SDO
	MINI_SHUTTLE_PIN_2_7 = 0x1D  # GPIO6
	MINI_SHUTTLE_PIN_2_8 = 0x1E  # GPIO7
\end{lstlisting}

\paragraph{SensorInterface}\label{SensorInterface}

To define Sensor interface.

\begin{lstlisting}
class SensorInterface(Enum):
	SPI = 0
	I2C = 1
\end{lstlisting}

\paragraph{I2CBus}\label{I2CBus}

Used to define the I2C type.

\begin{lstlisting}
	class I2CBus(Enum):
	BUS_I2C_0 = 0
	BUS_I2C_1 = 1
	BUS_I2C_MAX = 2
\end{lstlisting}

\paragraph{SPIBus}\label{SPIBus}

Used to define the SPI type.

\begin{lstlisting}
	class SPIBus(Enum):
	BUS_SPI_0 = 0
	BUS_SPI_1 = 1
	BUS_SPI_MAX = 2
\end{lstlisting}

\paragraph{PinInterruptMode}\label{PinInterruptMode}

Defines Pin interrupt modes.

\begin{lstlisting}
class PinInterruptMode(Enum):
	# Trigger interrupt on pin state change
	PIN_INTERRUPT_CHANGE = 0
	# Trigger interrupt when pin changes from low to high
	PIN_INTERRUPT_RISING_EDGE = 1
	# Trigger interrupt when pin changes from high to low
	PIN_INTERRUPT_FALLING_EDGE = 2
	PIN_INTERRUPT_MODE_MAXIMUM = 4
\end{lstlisting}

\paragraph{StreamingMode}\label{StreamingMode}

Streaming mode definitions

\begin{lstlisting}
class StreamingMode:
    STREAMING_MODE_POLLING = 0    # Polling mode streaming
    STREAMING_MODE_INTERRUPT = 1  # Interrupt mode streaming
\end{lstlisting}

\paragraph{StreamingState}\label{StreamingState}

Streaming state definitions

\begin{lstlisting}
class StreamingState:
	STREAMING_START = 1
	STREAMING_STOP = 0
\end{lstlisting}

\paragraph{SamplingUnits}\label{SamplingUnits}

Sampling Unit definitions

\begin{lstlisting}
class SamplingUnits:
    SAMPLING_TIME_IN_MICRO_SEC = 0x01  # sampling unit in micro second
    SAMPLING_TIME_IN_MILLI_SEC = 0x02  # sampling unit in milli second
\end{lstlisting}

